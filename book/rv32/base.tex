
\section{RV32I Base Instruction Set}
\index{RV32I}

\enote{Migrate all te details into the programming chapter and
reduce this section to the obligatory reference chapter.}%
\Gls{rv32}I refers to the basic 32-bit integer instructions.

%%%%%%%%%%%%%%%%%%%%%%%%%%%%%%%%%%%%%%%%%%%%%%%%%%%%%%%%%%%%%%%%%%%%%%%%
\subsection{LUI rd, imm}
\index{Instruction!LUI}

Load Upper Immediate.

\verb@rd@ $\leftarrow$ \verb@zr(imm)@

Copy the immediate value into bits 31:12 of the destination register and
place zeros into bits 11:0.
When XLEN is 64 or 128, the immediate value is sign-extended to the left.

\DrawInsnTypeUPicture{LUI t0, 3}{00000000000000000011001010110111}

\begin{verbatim}
00010074: 000032b7  lui     x5, 0x3         // x5 = 0x3000
 reg  0: 00000000 f0f0f0f0 f0f0f0f0 f0f0f0f0-f0f0f0f0 00003000 f0f0f0f0 f0f0f0f0
 reg  8: f0f0f0f0 f0f0f0f0 f0f0f0f0 f0f0f0f0-f0f0f0f0 f0f0f0f0 f0f0f0f0 f0f0f0f0
 reg 16: f0f0f0f0 f0f0f0f0 f0f0f0f0 f0f0f0f0-f0f0f0f0 f0f0f0f0 f0f0f0f0 f0f0f0f0
 reg 24: f0f0f0f0 f0f0f0f0 f0f0f0f0 f0f0f0f0-f0f0f0f0 f0f0f0f0 f0f0f0f0 f0f0f0f0
     pc: 00010078
\end{verbatim}

\DrawInsnTypeUPicture{LUI t0, 0xfffff}{11111111111111111111001010110111}

\begin{verbatim}
00010078: fffff2b7  lui     x5, 0xfffff         // x5 = 0xfffff000
 reg  0: 00000000 f0f0f0f0 f0f0f0f0 f0f0f0f0-f0f0f0f0 fffff000 f0f0f0f0 f0f0f0f0
 reg  8: f0f0f0f0 f0f0f0f0 f0f0f0f0 f0f0f0f0-f0f0f0f0 f0f0f0f0 f0f0f0f0 f0f0f0f0
 reg 16: f0f0f0f0 f0f0f0f0 f0f0f0f0 f0f0f0f0-f0f0f0f0 f0f0f0f0 f0f0f0f0 f0f0f0f0
 reg 24: f0f0f0f0 f0f0f0f0 f0f0f0f0 f0f0f0f0-f0f0f0f0 f0f0f0f0 f0f0f0f0 f0f0f0f0
     pc: 0001007c
\end{verbatim}



%Instruction Format and Example:
%
%\DrawInsnTypeUPicture{LUI t0, 3}{00000000000000000011001010110111}
%
%\begin{verbatim}
%00010074: 000032b7  lui     x5, 0x3         // x5 = 0x3000
% reg  0: 00000000 f0f0f0f0 f0f0f0f0 f0f0f0f0-f0f0f0f0 00003000 f0f0f0f0 f0f0f0f0
% reg  8: f0f0f0f0 f0f0f0f0 f0f0f0f0 f0f0f0f0-f0f0f0f0 f0f0f0f0 f0f0f0f0 f0f0f0f0
% reg 16: f0f0f0f0 f0f0f0f0 f0f0f0f0 f0f0f0f0-f0f0f0f0 f0f0f0f0 f0f0f0f0 f0f0f0f0
% reg 24: f0f0f0f0 f0f0f0f0 f0f0f0f0 f0f0f0f0-f0f0f0f0 f0f0f0f0 f0f0f0f0 f0f0f0f0
%     pc: 00010078
%\end{verbatim}
%
%\DrawInsnTypeUPicture{LUI t0, 0xfffff}{11111111111111111111001010110111}
%
%\begin{verbatim}
%00010078: fffff2b7  lui     x5, 0xfffff         // x5 = 0xfffff000
% reg  0: 00000000 f0f0f0f0 f0f0f0f0 f0f0f0f0-f0f0f0f0 fffff000 f0f0f0f0 f0f0f0f0
% reg  8: f0f0f0f0 f0f0f0f0 f0f0f0f0 f0f0f0f0-f0f0f0f0 f0f0f0f0 f0f0f0f0 f0f0f0f0
% reg 16: f0f0f0f0 f0f0f0f0 f0f0f0f0 f0f0f0f0-f0f0f0f0 f0f0f0f0 f0f0f0f0 f0f0f0f0
% reg 24: f0f0f0f0 f0f0f0f0 f0f0f0f0 f0f0f0f0-f0f0f0f0 f0f0f0f0 f0f0f0f0 f0f0f0f0
%     pc: 0001007c
%\end{verbatim}


%%%%%%%%%%%%%%%%%%%%%%%%%%%%%%%%%%%%%%%%%%%%%%%%%%%%%%%%%%%%%%%%%%%%%%%%
\subsection{AUIPC rd, imm}
\index{Instruction!AUIPC}

Add Upper Immediate to PC.

\verb@rd@ $\leftarrow$ \verb@pc + zr(imm)@

Create a signed 32-bit value by zero-extending imm[31:12] to the 
right (see \autoref{extension:zr}) and add this value to the 
\reg{pc} register, placing the result into \reg{rd}.

When XLEN is 64 or 128, the immediate value is also sign-extended 
to the left prior to being added to the \reg{pc} register.

\DrawInsnTypeUPicture{AUIPC t0, 3}{00000000000000000011001010110111}
\begin{verbatim}
0001007c: 00003297  auipc   x5, 0x3          // x5 = 0x1307c = 0x1007c + 0x3000
 reg  0: 00000000 f0f0f0f0 f0f0f0f0 f0f0f0f0-f0f0f0f0 0001307c f0f0f0f0 f0f0f0f0
 reg  8: f0f0f0f0 f0f0f0f0 f0f0f0f0 f0f0f0f0-f0f0f0f0 f0f0f0f0 f0f0f0f0 f0f0f0f0
 reg 16: f0f0f0f0 f0f0f0f0 f0f0f0f0 f0f0f0f0-f0f0f0f0 f0f0f0f0 f0f0f0f0 f0f0f0f0
 reg 24: f0f0f0f0 f0f0f0f0 f0f0f0f0 f0f0f0f0-f0f0f0f0 f0f0f0f0 f0f0f0f0 f0f0f0f0
     pc: 00010080
\end{verbatim}

\DrawInsnTypeUPicture{AUIPC t0, 0x81000}{10000001000000000000001010110111}
\begin{verbatim}
00010080: 81000297  auipc   x5, 0x81000          // x5 = 0x81010080 = 0x10080 + 0x81000000
 reg  0: 00000000 f0f0f0f0 f0f0f0f0 f0f0f0f0-f0f0f0f0 81010080 f0f0f0f0 f0f0f0f0
 reg  8: f0f0f0f0 f0f0f0f0 f0f0f0f0 f0f0f0f0-f0f0f0f0 f0f0f0f0 f0f0f0f0 f0f0f0f0
 reg 16: f0f0f0f0 f0f0f0f0 f0f0f0f0 f0f0f0f0-f0f0f0f0 f0f0f0f0 f0f0f0f0 f0f0f0f0
 reg 24: f0f0f0f0 f0f0f0f0 f0f0f0f0 f0f0f0f0-f0f0f0f0 f0f0f0f0 f0f0f0f0 f0f0f0f0
     pc: 00010084
\end{verbatim}


The AUIPC instruction supports two-instruction sequences to access arbitrary 
offsets from the PC for both control-flow transfers and data accesses. 
The combination of an AUIPC and the 12-bit immediate in a JALR can transfer 
control to any 32-bit PC-relative address, while an AUIPC plus the 12-bit 
immediate offset in regular load or store instructions can access any 32-bit 
PC-relative data address.~\cite[p.~14]{rvismv1v22:2017}


%%%%%%%%%%%%%%%%%%%%%%%%%%%%%%%%%%%%%%%%%%%%%%%%%%%%%%%%%%%%%%%%%%%%%%%%
\subsection{JAL rd, imm}
\index{Instruction!JAL}

Jump and link.

\verb@rd@ $\leftarrow$ \verb@pc + 4@\\
\verb@pc@ $\leftarrow$ \verb@pc + sx(imm<<1)@

This instruction saves the address of the next instruction
that would otherwise execute (located at \reg{pc}+4) into 
\reg{rd} and then adds immediate value to the \reg{pc} causing
an unconditional branch to take place.

The standard software conventions for calling subroutines
use \reg{x1} as the return address (\reg{rd} register in this 
case).~\cite[p.~16]{rvismv1v22:2017}


Encoding:

\DrawInsnTypeJPicture{JAL x7, .+16}{00000001000000000000001111101111}

imm demultiplexed value = $00000000000000001000_2 \ll 1 = 16_{10}$


State of registers before execution:

pc = 0x11114444

State of registers after execution:

pc = 0x11114454
x7 = 0x11114448

JAL provides a method to call a subroutine using a pc-relative address.


\DrawInsnTypeJPicture{JAL x7, .-16}{11111111000111111111001111101111}

imm demultiplexed value = $11111111111111111000_2 \ll 1 = -16_{10}$

State of registers before execution:

pc = 0x11114444

State of registers after execution:

pc = 0x11114434
x7 = 0x11114448


%%%%%%%%%%%%%%%%%%%%%%%%%%%%%%%%%%%%%%%%%%%%%%%%%%%%%%%%%%%%%%%%%%%%%%%%
\subsection{JALR rd, rs1, imm}
\index{Instruction!JALR}

Jump and link register.

\verb@rd@ $\leftarrow$ \verb@pc + 4@\\
\verb@pc@ $\leftarrow$ \verb@(rs1 + sx(imm)) & ~1@

This instruction saves the address of the next instruction
that would otherwise execute (located at \reg{pc}+4) into 
\reg{rd} and then adds the immediate value to the \reg{rs1} 
register and stores the sum into the \reg{pc} register causing
an unconditional branch to take place.

Note that the branch target address is calculated by 
sign-extending the imm[11:0] bits from the instruction, 
adding it to the \reg{rs1} register and {\em then} the 
LSB of the sum is to zero and the result is stored into the 
\reg{pc} register.
The discarding of the LSB allows the branch to refer to any 
even address.

The standard software conventions for calling subroutines
use \reg{x1} as the return address (\reg{rd} register in this 
case).~\cite[p.~16]{rvismv1v22:2017}


Encoding:

\DrawInsnTypeIPicture{JALR x1, x7, 4}{00000000010000111000000011100111}

Before:

pc = 0x11114444\\
x7 = 0x44444444

After

pc = 0x5555888c\\
x1 = 0x11114448

JALR provides a method to call a subroutine using a base-displacement address.

\DrawInsnTypeIPicture{JALR x1, x0, 5}{00000000010100000000000011100111}

Note that the least significant bit in the result of rs1+imm is 
discarded/set to zero before the result is saved in the pc.
 
pc = 0x11114444

After

pc = 0x00000004\\
x1 = 0x11114448



%%%%%%%%%%%%%%%%%%%%%%%%%%%%%%%%%%%%%%%%%%%%%%%%%%%%%%%%%%%%%%%%%%%%%%%%
\subsection{BEQ rs1, rs2, imm}
\index{Instruction!BEQ}

Branch if equal.

\verb@pc@ $\leftarrow$ \verb@(rs1 == rs2) ? pc+sx(imm[12:1]<<1) : pc+4@

Encoding:

\DrawInsnTypeBPicture{BEQ x3, x15, 2064}{00000000111100011000100011100011}

imm[12:1] = $010000001000_2 = 1032_{10}$\\
imm = $2064_{10}$\\
funct3 = $000_2$\\
rs1 = x3\\
rs2 = x15



%%%%%%%%%%%%%%%%%%%%%%%%%%%%%%%%%%%%%%%%%%%%%%%%%%%%%%%%%%%%%%%%%%%%%%%%
\subsection{BNE rs1, rs2, imm}
\index{Instruction!BNE}

Branch if Not Equal.

\verb@pc@ $\leftarrow$ \verb@(rs1 != rs2) ? pc+sx(imm[12:1]<<1) : pc+4@

Encoding:

\DrawInsnTypeBPicture{BNE x3, x15, 2064}{00000000111100011001100011100011}

imm[12:1] = $010000001000_2 = 1032_{10}$\\
imm = $2064_{10}$\\
funct3 = $001_2$\\
rs1 = x3\\
rs2 = x15


%%%%%%%%%%%%%%%%%%%%%%%%%%%%%%%%%%%%%%%%%%%%%%%%%%%%%%%%%%%%%%%%%%%%%%%%
\subsection{BLT rs1, rs2, imm}
\index{Instruction!BLT}

Branch if Less Than.

\verb@pc@ $\leftarrow$ \verb@(rs1 < rs2) ? pc+sx(imm[12:1]<<1) : pc+4@

Encoding:

\DrawInsnTypeBPicture{BLT x3, x15, 2064}{00000000111100011100100011100011}

imm[12:1] = $010000001000_2 = 1032_{10}$\\
imm = $2064_{10}$\\
funct3 = $100_2$\\
rs1 = x3\\
rs2 = x15



%%%%%%%%%%%%%%%%%%%%%%%%%%%%%%%%%%%%%%%%%%%%%%%%%%%%%%%%%%%%%%%%%%%%%%%%
\subsection{BGE rs1, rs2, imm}
\index{Instruction!BGE}

Branch if Greater or Equal.

\verb@pc@ $\leftarrow$ \verb@(rs1 >= rs2) ? pc+sx(imm[12:1]<<1) : pc+4@

Encoding:

\DrawInsnTypeBPicture{BGE x3, x15, 2064}{00000000111100011101100011100011}

imm[12:1] = $010000001000_2 = 1032_{10}$\\
imm = $2064_{10}$\\
funct3 = $101_2$\\
rs1 = x3\\
rs2 = x15

%%%%%%%%%%%%%%%%%%%%%%%%%%%%%%%%%%%%%%%%%%%%%%%%%%%%%%%%%%%%%%%%%%%%%%%%
\subsection{BLTU rs1, rs2, imm}
\index{Instruction!BLTU}

Branch if Less Than Unsigned.

\verb@pc@ $\leftarrow$ \verb@(rs1 < rs2) ? pc+sx(imm[12:1]<<1) : pc+4@

Encoding:

\DrawInsnTypeBPicture{BLTU x3, x15, 2064}{00000000111100011110100011100011}

imm[12:1] = $010000001000_2 = 1032_{10}$\\
imm = $2064_{10}$\\
funct3 = $110_2$\\
rs1 = x3\\
rs2 = x15


%%%%%%%%%%%%%%%%%%%%%%%%%%%%%%%%%%%%%%%%%%%%%%%%%%%%%%%%%%%%%%%%%%%%%%%%
\subsection{BGEU rs1, rs2, imm}
\index{Instruction!BGEU}

Branch if Greater or Equal Unsigned.

\verb@pc@ $\leftarrow$ \verb@(rs1 >= rs2) ? pc+sx(imm[12:1]<<1) : pc+4@

Encoding:

\DrawInsnTypeBPicture{BGEU x3, x15, 2064}{00000000111100011111100011100011}
\enote{use symbols in branch examples}

imm[12:1] = $010000001000_2 = 1032_{10}$\\
imm = $2064_{10}$\\
funct3 = $111_2$\\
rs1 = x3\\
rs2 = x15

%%%%%%%%%%%%%%%%%%%%%%%%%%%%%%%%%%%%%%%%%%%%%%%%%%%%%%%%%%%%%%%%%%%%%%%%
\subsection{LB rd, imm(rs1)}
\index{Instruction!LB}

Load byte.

\verb@rd@ $\leftarrow$ \verb@sx(m8(rs1+sx(imm)))@\\
\verb@pc@ $\leftarrow$ \verb@pc+4@

Load an 8-bit value from memory at address \verb@rs1+imm@, then 
sign-extend it to 32 bits before storing it in \verb@rd@


Encoding:

\DrawInsnTypeIPicture{LB x7, 4(x3)}{00000000010000011000001110000011}

%%%%%%%%%%%%%%%%%%%%%%%%%%%%%%%%%%%%%%%%%%%%%%%%%%%%%%%%%%%%%%%%%%%%%%%%
\subsection{LH rd, imm(rs1)}
\index{Instruction!LH}

Load halfword.

\verb@rd@ $\leftarrow$ \verb@sx(m16(rs1+sx(imm)))@\\
\verb@pc@ $\leftarrow$ \verb@pc+4@

Load a 16-bit value from memory at address \verb@rs1+imm@, then 
sign-extend it to 32 bits before storing it in \verb@rd@


Encoding:

\DrawInsnTypeIPicture{LH x7, 4(x3)}{00000000010000011001001110000011}

%%%%%%%%%%%%%%%%%%%%%%%%%%%%%%%%%%%%%%%%%%%%%%%%%%%%%%%%%%%%%%%%%%%%%%%%
\subsection{LW rd, imm(rs1)}
\index{Instruction!LW}

Load word.

\verb@rd@ $\leftarrow$ \verb@sx(m32(rs1+sx(imm)))@\\
\verb@pc@ $\leftarrow$ \verb@pc+4@

Load a 32-bit value from memory at address \verb@rs1+imm@, then 
store it in \verb@rd@

Encoding:

\DrawInsnTypeIPicture{LW x7, 4(x3)}{00000000010000011010001110000011}


%%%%%%%%%%%%%%%%%%%%%%%%%%%%%%%%%%%%%%%%%%%%%%%%%%%%%%%%%%%%%%%%%%%%%%%%
\subsection{LBU rd, imm(rs1)}
\index{Instruction!LBU}

Load byte unsigned.

\verb@rd@ $\leftarrow$ \verb@zx(m8(rs1+sx(imm)))@\\
\verb@pc@ $\leftarrow$ \verb@pc+4@

Load an 8-bit value from memory at address \verb@rs1+imm@, then 
zero-extend it to 32 bits before storing it in \verb@rd@

Encoding:

\DrawInsnTypeIPicture{LBU x7, 4(x3)}{00000000010000011100001110000011}

%%%%%%%%%%%%%%%%%%%%%%%%%%%%%%%%%%%%%%%%%%%%%%%%%%%%%%%%%%%%%%%%%%%%%%%%
\subsection{LHU rd, imm(rs1)}
\index{Instruction!LHU}

Load halfword unsigned.

\verb@rd@ $\leftarrow$ \verb@zx(m16(rs1+sx(imm)))@\\
\verb@pc@ $\leftarrow$ \verb@pc+4@

Load an 16-bit value from memory at address \verb@rs1+imm@, then 
zero-extend it to 32 bits before storing it in \verb@rd@

Encoding:

\DrawInsnTypeIPicture{LHU x7, 4(x3)}{00000000010000011101001110000011}

%%%%%%%%%%%%%%%%%%%%%%%%%%%%%%%%%%%%%%%%%%%%%%%%%%%%%%%%%%%%%%%%%%%%%%%%
\subsection{SB rs2, imm(rs1)}
\index{Instruction!SB}

Store Byte.

\verb@m8(rs1+sx(imm))@ $\leftarrow$ \verb@rs2[7:0]@\\
\verb@pc@ $\leftarrow$ \verb@pc+4@

Store the 8-bit value in \verb@rs2[7:0]@ into memory at 
address \verb@rs1+imm@.

Encoding:

\DrawInsnTypeSPicture{SB x3, 19(x15)}{00000000111100011000100110100011}

%%%%%%%%%%%%%%%%%%%%%%%%%%%%%%%%%%%%%%%%%%%%%%%%%%%%%%%%%%%%%%%%%%%%%%%%
\subsection{SH rs2, imm(rs1)}
\index{Instruction!SH}

Store Halfword.

\verb@m16(rs1+sx(imm))@ $\leftarrow$ \verb@rs2[15:0]@\\
\verb@pc@ $\leftarrow$ \verb@pc+4@

Store the 16-bit value in \verb@rs2[15:0]@ into memory at 
address \verb@rs1+imm@.


Encoding:

\DrawInsnTypeSPicture{SH x3, 19(x15)}{00000000111100011001100110100011}

%%%%%%%%%%%%%%%%%%%%%%%%%%%%%%%%%%%%%%%%%%%%%%%%%%%%%%%%%%%%%%%%%%%%%%%%
\subsection{SW rs2, imm(rs1)}
\index{Instruction!SW}

Store Word

\verb@m16(rs1+sx(imm))@ $\leftarrow$ \verb@rs2[31:0]@\\
\verb@pc@ $\leftarrow$ \verb@pc+4@

Store the 32-bit value in \verb@rs2@ into memory at address \verb@rs1+imm@.

Encoding:

\DrawInsnTypeSPicture{SW x3, 19(x15)}{00000000111100011010100110100011}

Show pos \& neg imm examples.

%%%%%%%%%%%%%%%%%%%%%%%%%%%%%%%%%%%%%%%%%%%%%%%%%%%%%%%%%%%%%%%%%%%%%%%%
\subsection{ADDI rd, rs1, imm}
\index{Instruction!ADDI}

Add Immediate

\verb@rd@ $\leftarrow$ \verb@rs1+sx(imm)@\\
\verb@pc@ $\leftarrow$ \verb@pc+4@

Encoding:

\DrawInsnTypeIPicture{ADDI x1, x7, 4}{00000000010000111000000010010011}

Before:

x7 = 0x11111111

After:

x1 = 0x11111115

%%%%%%%%%%%%%%%%%%%%%%%%%%%%%%%%%%%%%%%%%%%%%%%%%%%%%%%%%%%%%%%%%%%%%%%%
\subsection{SLTI rd, rs1, imm}
\index{Instruction!SLTI}

Set LessThan Immediate

\verb@rd@ $\leftarrow$ \verb@(rs1 < sx(imm)) ? 1 : 0@\\
\verb@pc@ $\leftarrow$ \verb@pc+4@

If the sign-extended immediate value is less than the value
in the \reg{rs1} register then the value 1 is stored in the 
\reg{rd} register.  Otherwise the value 0 is stored in the
\reg{rd} register. 

Encoding:

\DrawInsnTypeIPicture{SLTI x1, x7, 4}{00000000010000111010000010010011}

Before:

x7 = 0x11111111

After:

x1 = 0x00000000

%%%%%%%%%%%%%%%%%%%%%%%%%%%%%%%%%%%%%%%%%%%%%%%%%%%%%%%%%%%%%%%%%%%%%%%%
\subsection{SLTIU rd, rs1, imm}
\index{Instruction!SLTIU}

Set LessThan Immediate Unsigned

\verb@rd@ $\leftarrow$ \verb@(rs1 < sx(imm)) ? 1 : 0@\\
\verb@pc@ $\leftarrow$ \verb@pc+4@

If the sign-extended immediate value is less than the value
in the \reg{rs1} register then the value 1 is stored in the 
\reg{rd} register.  Otherwise the value 0 is stored in the
\reg{rd} register.  Both the immediate and \reg{rs1} register
values are treated as unsigned numbers for the purposes of the 
comparison.\footnote{The immediate value is first sign-extended to
XLEN bits then treated as an unsigned number.\cite[p.~14]{rvismv1v22:2017}}


Encoding:

\DrawInsnTypeIPicture{SLTIU x1, x7, 4}{00000000010000111011000010010011}

Before:

x7 = 0x81111111

After:

x1 = 0x00000001

%%%%%%%%%%%%%%%%%%%%%%%%%%%%%%%%%%%%%%%%%%%%%%%%%%%%%%%%%%%%%%%%%%%%%%%%
\subsection{XORI rd, rs1, imm}
\index{Instruction!XORI}

Exclusive Or Immediate

\verb@rd@ $\leftarrow$ \verb@rs1 ^ sx(imm)@\\
\verb@pc@ $\leftarrow$ \verb@pc+4@

The logical XOR of the sign-extended immediate value and the value 
in the \reg{rs1} register is stored in the \reg{rd} register.

Encoding:

\DrawInsnTypeIPicture{XORI x1, x7, 4}{00000000010000111100000010010011}

Before:

x7 = 0x81111111

After:

x1 = 0x81111115

%%%%%%%%%%%%%%%%%%%%%%%%%%%%%%%%%%%%%%%%%%%%%%%%%%%%%%%%%%%%%%%%%%%%%%%%
\subsection{ORI rd, rs1, imm}
\index{Instruction!ORI}

Or Immediate

\verb@rd@ $\leftarrow$ \verb@rs1 | sx(imm)@\\
\verb@pc@ $\leftarrow$ \verb@pc+4@

The logical OR of the sign-extended immediate value and the value 
in the \reg{rs1} register is stored in the \reg{rd} register.

Encoding:

\DrawInsnTypeIPicture{ORI x1, x7, 4}{00000000010000111110000010010011}

Before:

x7 = 0x81111111

After:

x1 = 0x81111115

%%%%%%%%%%%%%%%%%%%%%%%%%%%%%%%%%%%%%%%%%%%%%%%%%%%%%%%%%%%%%%%%%%%%%%%%
\subsection{ANDI rd, rs1, imm}
\index{Instruction!ANDI}

And Immediate

\verb@rd@ $\leftarrow$ \verb@rs1 & sx(imm)@\\
\verb@pc@ $\leftarrow$ \verb@pc+4@

The logical AND of the sign-extended immediate value and the value 
in the \reg{rs1} register is stored in the \reg{rd} register.


Encoding:

\DrawInsnTypeIPicture{ANDI x1, x7, 4}{00000000010000111111000010010011}

Before:

x7 = 0x81111111

After:

x1 = 0x81111115

%%%%%%%%%%%%%%%%%%%%%%%%%%%%%%%%%%%%%%%%%%%%%%%%%%%%%%%%%%%%%%%%%%%%%%%%
\subsection{SLLI rd, rs1, shamt}
\index{Instruction!SLLI}

Shift Left Logical Immediate

\verb@rd@ $\leftarrow$ \verb@rs1 << shamt@\\
\verb@pc@ $\leftarrow$ \verb@pc+4@


SLLI is a logical left shift operation (zeros are shifted
into the lower bits).  The value in rs1 shifted left shamt
number of bits and the result placed into rd.~\cite[p.~14]{rvismv1v22:2017}

Encoding:

\DrawInsnTypeRShiftPicture{SLLI x7, x3, 2}{00000000001000011001001110100011}

x3 = 0x81111111

After:

x7 = 0x04444444

%%%%%%%%%%%%%%%%%%%%%%%%%%%%%%%%%%%%%%%%%%%%%%%%%%%%%%%%%%%%%%%%%%%%%%%%
\subsection{SRLI rd, rs1, shamt}
\index{Instruction!SRLI}

Shift Right Logical Immediate

\verb@rd@ $\leftarrow$ \verb@rs1 >> shamt@\\
\verb@pc@ $\leftarrow$ \verb@pc+4@

SRLI is a logical right shift operation (zeros are shifted
into the higher bits).  The value in rs1 shifted right shamt
number of bits and the result placed into rd.~\cite[p.~14]{rvismv1v22:2017}


Encoding:

\DrawInsnTypeRShiftPicture{SRLI x7, x3, 2}{00000000001000011101001110010011}

x3 = 0x81111111

After:

x7 = 0x20444444

%%%%%%%%%%%%%%%%%%%%%%%%%%%%%%%%%%%%%%%%%%%%%%%%%%%%%%%%%%%%%%%%%%%%%%%%
\subsection{SRAI rd, rs1, shamt}
\index{Instruction!SRAI}

Shift Right Arithmetic Immediate

\verb@rd@ $\leftarrow$ \verb@rs1 >> shamt@\\
\verb@pc@ $\leftarrow$ \verb@pc+4@

SRAI is a logical right shift operation (zeros are shifted
into the higher bits).  The value in rs1 shifted right shamt
number of bits and the result placed into rd.~\cite[p.~14]{rvismv1v22:2017}

Encoding:

\DrawInsnTypeRShiftPicture{SRAI x7, x3, 2}{01000000001000011101001110010011}

x3 = 0x81111111

After:

x7 = 0xe0444444


%%%%%%%%%%%%%%%%%%%%%%%%%%%%%%%%%%%%%%%%%%%%%%%%%%%%%%%%%%%%%%%%%%%%%%%%
\subsection{ADD rd, rs1, rs2}
\index{Instruction!ADD}

Add 

\verb@rd@ $\leftarrow$ \verb@rs1 + rs2@\\
\verb@pc@ $\leftarrow$ \verb@pc+4@

ADD performs addition. Overflows are ignored and the low 32 bits of 
the result are written to rd.~\cite[p.~15]{rvismv1v22:2017}

Encoding:

\DrawInsnTypeRPicture{ADD x7, x3, x31}{00000001111100011000001110110011}

x3  = 0x81111111
x31 = 0x22222222

After:

x7 = 0xa3333333

%%%%%%%%%%%%%%%%%%%%%%%%%%%%%%%%%%%%%%%%%%%%%%%%%%%%%%%%%%%%%%%%%%%%%%%%
\subsection{SUB rd, rs1, rs2}
\index{Instruction!SUB}

Subtract

\verb@rd@ $\leftarrow$ \verb@rs1 - rs2@\\
\verb@pc@ $\leftarrow$ \verb@pc+4@

SUB performs subtraction. Underflows are ignored and the low 32 bits of 
the result are written to rd.~\cite[p.~15]{rvismv1v22:2017}

Encoding:

\DrawInsnTypeRPicture{SUB x7, x3, x31}{01000001111100011000001110110011}

x3  = 0x83333333
x31 = 0x01111111

After:

x7 = 0x82222222


%%%%%%%%%%%%%%%%%%%%%%%%%%%%%%%%%%%%%%%%%%%%%%%%%%%%%%%%%%%%%%%%%%%%%%%%
\subsection{SLL rd, rs1, rs2}
\index{Instruction!SLL}

Shift Left Logical

\verb@rd@ $\leftarrow$ \verb@rs1 << rs2@\\
\verb@pc@ $\leftarrow$ \verb@pc+4@

SLL performs a logical left shift on the value in register rs1 by 
the shift amount held in the lower 5 bits of register rs2.~\cite[p.~15]{rvismv1v22:2017}

Encoding:

\DrawInsnTypeRPicture{SLL x7, x3, x31}{00000001111100011001001110110011}

x3  = 0x83333333\\
x31 = 0x00000002

After:

x7 = 0x0ccccccc


%%%%%%%%%%%%%%%%%%%%%%%%%%%%%%%%%%%%%%%%%%%%%%%%%%%%%%%%%%%%%%%%%%%%%%%%
\subsection{SLT rd, rs1, rs2}
\index{Instruction!SLT}

Set Less Than

\verb@rd@ $\leftarrow$ \verb@(rs1 < rs2) ? 1 : 0@\\
\verb@pc@ $\leftarrow$ \verb@pc+4@

SLT performs a signed compare, writing 1 to \reg{rd} if \reg{rs1} $<$ \reg{rs2}, 0 
otherwise.~\cite[p.~15]{rvismv1v22:2017}

Encoding:

\DrawInsnTypeRPicture{SLT x7, x3, x31}{00000001111100011010001110110011}

x3  = 0x83333333\\
x31 = 0x00000002

After:

x7 = 0x00000001

%%%%%%%%%%%%%%%%%%%%%%%%%%%%%%%%%%%%%%%%%%%%%%%%%%%%%%%%%%%%%%%%%%%%%%%%
\subsection{SLTU rd, rs1, rs2}
\index{Instruction!SLTU}

Set Less Than Unsigned 

\verb@rd@ $\leftarrow$ \verb@(rs1 < rs2) ? 1 : 0@\\
\verb@pc@ $\leftarrow$ \verb@pc+4@

SLTU performs an unsigned compare, writing 1 to \reg{rd} if \reg{rs1} $<$ \reg{rs2}, 0 otherwise. 
Note, SLTU rd, x0, rs2 sets \reg{rd} to 1 if \reg{rs2} is not equal to zero, otherwise 
sets \reg{rd} to zero (assembler pseudoinstruction \verb@SNEZ rd, rs@).~\cite[p.~15]{rvismv1v22:2017}

Encoding:

\DrawInsnTypeRPicture{SLTU x7, x3, x31}{00000001111100011011001110110011}

x3  = 0x83333333\\
x31 = 0x00000002

After:

x7 = 0x00000000

%%%%%%%%%%%%%%%%%%%%%%%%%%%%%%%%%%%%%%%%%%%%%%%%%%%%%%%%%%%%%%%%%%%%%%%%
\subsection{XOR rd, rs1, rs2}
\index{Instruction!XOR}

Exclusive Or

\verb@rd@ $\leftarrow$ \verb@rs1 ^ rs2@\\
\verb@pc@ $\leftarrow$ \verb@pc+4@

XOR performs a bit-wise exclusive or on rs1 and rs2.  
The result is stored on rd.

Encoding:

\DrawInsnTypeRPicture{XOR x7, x3, x31}{00000001111100011100001110110011}

x3  = 0x83333333\\
x31 = 0x1888ffff

After:

x7 = 0x9bbbcccc

%%%%%%%%%%%%%%%%%%%%%%%%%%%%%%%%%%%%%%%%%%%%%%%%%%%%%%%%%%%%%%%%%%%%%%%%
\subsection{SRL rd, rs1, rs2}
\index{Instruction!SRL}

Shift Right Logical

\verb@rd@ $\leftarrow$ \verb@rs1 >> rs2@\\
\verb@pc@ $\leftarrow$ \verb@pc+4@

SRL performs a logical right shift on the value in register rs1 by 
the shift amount held in the lower 5 bits of 
register rs2.~\cite[p.~15]{rvismv1v22:2017}

Encoding:

\DrawInsnTypeRPicture{SRL x7, x3, x31}{00000001111100011101001110110011}

x3  = 0x83333333\\
x31 = 0x00000010

After:

x7 = 0x00008333

%%%%%%%%%%%%%%%%%%%%%%%%%%%%%%%%%%%%%%%%%%%%%%%%%%%%%%%%%%%%%%%%%%%%%%%%
\subsection{SRA rd, rs1, rs2}
\index{Instruction!SRA}

Shift Right Arithmetic

\verb@rd@ $\leftarrow$ \verb@rs1 >> rs2@\\
\verb@pc@ $\leftarrow$ \verb@pc+4@

SRA performs an arithmetic right shift (the original sign bit is copied 
into the vacated upper bits) on the value in register rs1 by the shift 
amount held in the lower 5 bits of 
register rs2.~\cite[p.~14,~15]{rvismv1v22:2017}

Encoding:

\DrawInsnTypeRPicture{SRA x7, x3, x31}{01000001111100011101001110110011}

x3  = 0x83333333\\
x31 = 0x00000010

After:

x7 = 0xffff8333

%%%%%%%%%%%%%%%%%%%%%%%%%%%%%%%%%%%%%%%%%%%%%%%%%%%%%%%%%%%%%%%%%%%%%%%%
\subsection{OR rd, rs1, rs2}
\index{Instruction!OR}

Or 

\verb@rd@ $\leftarrow$ \verb@rs1 | rs2@\\
\verb@pc@ $\leftarrow$ \verb@pc+4@

OR is a logical operation that performs a bit-wise OR on 
register rs1 and rs2 and then places the result 
in rd.~\cite[p.~14]{rvismv1v22:2017}

Encoding:

\DrawInsnTypeRPicture{OR x7, x3, x31}{00000001111100011101001110110011}

x3  = 0x83333333\\
x31 = 0x00000440

After:

x7 = 0x83333773

%%%%%%%%%%%%%%%%%%%%%%%%%%%%%%%%%%%%%%%%%%%%%%%%%%%%%%%%%%%%%%%%%%%%%%%%
\subsection{AND rd, rs1, rs2}
\index{Instruction!AND}

And

\verb@rd@ $\leftarrow$ \verb@rs1 & rs2@\\
\verb@pc@ $\leftarrow$ \verb@pc+4@

AND is a logical operation that performs a bit-wise AND on 
register rs1 and rs2 and then places the result 
in rd.~\cite[p.~14]{rvismv1v22:2017}

Encoding:

\DrawInsnTypeRPicture{AND x7, x3, x31}{00000001111100011110001110110011}

x3  = 0x83333333\\
x31 = 0x00000fe2

After:

x7 = 0x00000322

%%%%%%%%%%%%%%%%%%%%%%%%%%%%%%%%%%%%%%%%%%%%%%%%%%%%%%%%%%%%%%%%%%%%%%%%
\subsection{FENCE predecessor, successor}
\index{Instruction!FENCE}

\enote{Which of the i, o, r and w goes into each bit?  See what gas does.}%
The FENCE instruction is used to order device I/O and memory accesses as 
viewed by other RISC-V harts and external devices or co-processors. Any 
combination of device input (I), device output (O), memory reads (R), 
and memory writes (W) may be ordered with respect to any combination
of the same. Informally, no other RISC-V hart or external device can 
observe any operation in the successor set following a FENCE before any 
operation in the predecessor set preceding the FENCE. The execution 
environment will define what I/O operations are possible, and in particular, 
which load and store instructions might be treated and ordered as device 
input and device output operations respectively rather than memory reads 
and writes. For example, memory-mapped I/O devices will typically be 
accessed with uncached loads and stores that are ordered using the I and O
 bits rather than the R and W bits. Instruction-set extensions might 
also describe new coprocessor I/O instructions that will also be ordered 
using the I and O bits in a FENCE.~\cite[p.~21]{rvismv1v22:2017}

Operation:

\verb@pc@ $\leftarrow$ \verb@pc+4@

Encoding:

\DrawInsnTypeFPicture{FENCE iorw, iorw}{00001111111100000000000000001111}

%%%%%%%%%%%%%%%%%%%%%%%%%%%%%%%%%%%%%%%%%%%%%%%%%%%%%%%%%%%%%%%%%%%%%%%%
\subsection{FENCE.I}
\index{Instruction!FENCE.I}

The FENCE.I instruction is used to synchronize the instruction and 
data streams. RISC-V does not guarantee that stores to instruction 
memory will be made visible to instruction fetches on the same
RISC-V hart until a FENCE.I instruction is executed. A FENCE.I 
instruction only ensures that a subsequent instruction fetch on 
a RISC-V hart will see any previous data stores already visible to 
the same RISC-V hart. FENCE.I does not ensure that other RISC-V harts' 
instruction fetches will observe the local hart's stores in a 
multiprocessor system. To make a store to instruction memory
visible to all RISC-V harts, the writing hart has to execute a 
data FENCE before requesting that all remote RISC-V harts execute 
a FENCE.I.~\cite[p.~21]{rvismv1v22:2017}

Operation:

\verb@pc@ $\leftarrow$ \verb@pc+4@

Encoding:

\DrawInsnTypeFPicture{FENCE.I}{00000000000000000001000000001111}


%%%%%%%%%%%%%%%%%%%%%%%%%%%%%%%%%%%%%%%%%%%%%%%%%%%%%%%%%%%%%%%%%%%%%%%%
\subsection{ECALL}
\index{Instruction!ECALL}

The ECALL instruction is used to make a request to the supporting 
execution environment, which is usually an operating system. The ABI 
for the system will define how parameters for the environment
request are passed, but usually these will be in defined locations 
in the integer register file.~\cite[p.~24]{rvismv1v22:2017}

\DrawInsnTypeEPicture{ECALL}{00000000000000000000000001110011}


%%%%%%%%%%%%%%%%%%%%%%%%%%%%%%%%%%%%%%%%%%%%%%%%%%%%%%%%%%%%%%%%%%%%%%%%
\subsection{EBREAK}
\index{Instruction!EBREAK}

The EBREAK instruction is used by debuggers to cause control to be 
transferred back to a debugging environment.~\cite[p.~24]{rvismv1v22:2017}

\DrawInsnTypeEPicture{EBREAK}{00000000000100000000000001110011}


